%%%%%%%%%%%%%%%%%%%%%%%%%%%%%%%%%%%%%%%%%%%%%%%%%%%%%%%%%%%%%%%%%%%%%%%%%%
%%		Documento para “feedback” dos participantes de eventos			%%
%%	produzidos pelo sancaLUG. Baseado no documento de mesmo propósito	%%
%%	desenvolvido pelo grupo tasLUG, para participantes do “Software		%%
%%	Freedom Day” do ano de 2010.										%%
%%		O código fonte usado para fazer esse formulário foi				%%
%%	aproveitado de um documento disponível no site:						%%
%%	http://www.svenhartenstein.de/creating-questionnaires-with-latex/	%%
%%%%%%%%%%%%%%%%%%%%%%%%%%%%%%%%%%%%%%%%%%%%%%%%%%%%%%%%%%%%%%%%%%%%%%%%%%

\documentclass[a4paper,11pt,oneside,headsepline]{scrartcl}
\usepackage{polyglossia}
	\setdefaultlanguage{brazil}
\usepackage{microtype}
\usepackage{fontawesome}	% usado para o TUX
\usepackage{showframe}		% usado para mostrar as margens
\usepackage{hyperref}		% usado para links
\usepackage{wasysym}		% provides \ocircle and \Box
\usepackage{enumitem}		% easy control of topsep and leftmargin for lists
%\usepackage{color}			% used for background color
\usepackage{forloop}		% used for \Qrating and \Qlines
\usepackage{ifthen}			% used for \Qitem and \QItem
\usepackage{typearea}
\areaset{17cm}{26cm}
\setlength{\topmargin}{-1cm}
\usepackage{scrpage2}
\pagestyle{scrheadings}
\ihead{\emph{Feedback Installfest} \textbf{sancaLUG}}
\ohead{\pagemark}
\chead{}
\cfoot{}

%%%%%%%%%%%%%%%%%%%%%%%%%%%%%%%%%%%%%%%%%%%%%%%%%%%%%%%%%%%%
%% Beginning of questionnaire command definitions %%
%%%%%%%%%%%%%%%%%%%%%%%%%%%%%%%%%%%%%%%%%%%%%%%%%%%%%%%%%%%%
%%
%% 2010, 2012 by Sven Hartenstein
%% mail@svenhartenstein.de
%% http://www.svenhartenstein.de
%%
%% Please be warned that this is NOT a full-featured framework for
%% creating (all sorts of) questionnaires. Rather, it is a small
%% collection of LaTeX commands that I found useful when creating a
%% questionnaire. Feel free to copy and adjust any parts you like.
%% Most probably, you will want to change the commands, so that they
%% fit your taste.
%%
%% Also note that I am not a LaTeX expert! Things can very likely be
%% done much more elegant than I was able to. If you have suggestions
%% about what can be improved please send me an email. I intend to
%% add good tipps to my website and to name contributers of course.
%%
%% 10/2012: Thanks to karathan for the suggestion to put \noindent
%% before \rule!

%% \Qq = Questionaire question. Oh, this is just too simple. It helps
%% making it easy to globally change the appearance of questions.
\newcommand{\Qq}[1]{\textbf{#1}}

%% \QO = Circle or box to be ticked. Used both by direct call and by
%% \Qrating and \Qlist.
%\newcommand{\QO}{$\Box$}% or: $\ocircle$
\newcommand{\QO}{$\ocircle$}

%% \Qrating = Automatically create a rating scale with NUM steps, like
%% this: 0--0--0--0--0.
\newcounter{qr}
\newcommand{\Qrating}[1]{\QO\forloop{qr}{1}{\value{qr} < #1}{---\QO}}

%% \Qline = Again, this is very simple. It helps setting the line
%% thickness globally. Used both by direct call and by \Qlines.
\newcommand{\Qline}[1]{\noindent\rule{#1}{0.6pt}}

%% \Qlines = Insert NUM lines with width=\linewith. You can change the
%% \vskip value to adjust the spacing.
\newcounter{ql}
\newcommand{\Qlines}[1]{\forloop{ql}{0}{\value{ql}<#1}{\vskip0em\Qline{\linewidth}}}

%% \Qlist = This is an environment very similar to itemize but with
%% \QO in front of each list item. Useful for classical multiple
%% choice. Change leftmargin and topsep accourding to your taste.
\newenvironment{Qlist}{%
\renewcommand{\labelitemi}{\QO}
\begin{itemize}[leftmargin=1.5em,topsep=-.5em]
}{%
\end{itemize}
}

%% \Qtab = A "tabulator simulation". The first argument is the
%% distance from the left margin. The second argument is content which
%% is indented within the current row.
\newlength{\qt}
\newcommand{\Qtab}[2]{
\setlength{\qt}{\linewidth}
\addtolength{\qt}{-#1}
\hfill\parbox[t]{\qt}{\raggedright #2}
}

%% \Qitem = Item with automatic numbering. The first optional argument
%% can be used to create sub-items like 2a, 2b, 2c, ... The item
%% number is increased if the first argument is omitted or equals 'a'.
%% You will have to adjust this if you prefer a different numbering
%% scheme. Adjust topsep and leftmargin as needed.
\newcounter{itemnummer}
\newcommand{\Qitem}[2][]{% #1 optional, #2 notwendig
\ifthenelse{\equal{#1}{}}{\stepcounter{itemnummer}}{}
\ifthenelse{\equal{#1}{a}}{\stepcounter{itemnummer}}{}
\begin{enumerate}[topsep=2pt,leftmargin=2.8em]
\item[\textbf{\arabic{itemnummer}#1.}] #2
\end{enumerate}
}

%%%%%%%%%%%%%%%%%%%%%%%%%%%%%%%%%%%%%%%%%%%%%%%%%%%%%%%%%%%%
%% End of questionnaire command definitions %%
%%%%%%%%%%%%%%%%%%%%%%%%%%%%%%%%%%%%%%%%%%%%%%%%%%%%%%%%%%%%

\begin{document}

\begin{center}
\textbf{\huge \emph{Feedback Installfest} sancaLUG\faLinux}
\end{center}\vskip1em

\noindent O grupo \textbf{sancaLUG} (São Carlos e região Linux
User Group) gostaria de agradecer por participar deste evento.
Esperamos que hoje você tenha conhecido um pouco do que o grupo
está disposto a fazer pela comunidade.

Com essa pesquisa pretendemos melhorar os nosssos serviços prestados
voluntariamente à comunidade.

\section*{Questionário}

\Qitem{ \Qq{Como ficou sabendo sobre o evento?}
	\begin{Qlist}
		\item Por Amigos
		\item Por banners ou fliers
		\item Diretamente por organizadores
		\item Pelas páginas do sancaLUG
		\item Pelas páginas da USP
		\vskip .3em
		Outro: \Qline{14cm}
	\end{Qlist}
}

\Qitem{ \Qq{Você já tinha ouvido falar no grupo \textbf{sancaLUG}, de onde?}
	\begin{Qlist}
		\item Não
		\item Sim, por algum outro evento organizado por eles
		\item Sim, por um amigo
		\item Sim, pelas páginas do sancaLUG
		\vskip .3em
		Outro: \Qline{14cm}
	\end{Qlist}
}

\Qitem{ \Qq{Atualmente você utiliza Linux na sua casa ou trabalho?}
	\begin{Qlist}
		\item Sim. Qual: \Qline{5cm}
		\item Não
	\end{Qlist}
}

\dotfill

\newpage

\Qitem{ \Qq{O que o fez participar dessa \emph{installfest}?}
	\begin{Qlist}
		\item Instalar uma distro em sua máquina
		\item Conhecer as distros Linux
		\item Conhecer o evento, o grupo ou as pessoas participantes
		\item Ajudar nas instalações
	\end{Qlist}
}

\Qitem{ \Qq{O que achou do evento?}
	\begin{Qlist}
		\item Ótimo
		\item Bom
		\item Regular
		\item Ruim
	\end{Qlist}
}

\Qitem{ \Qq{Deixe sua sugestão e seu \emph{e-mail} para contato!}
		\Qlines{3}
}

\dotfill

\noindent Novamente agradecemos a sua presença,

Esperamos que tenha aproveitado a \emph{Installfest} promovida pelo
\textbf{sancaLUG} e participe dos nossos próximos eventos. Você pode
ficar por dentro das novidades pela nossa página do facebook
(\url{fb.com/sancaLUG}) ou pela wiki (\url{sancaLUG.org}).

%%%
%TODO:
%	não esquecer de tirar o showframe
%	formular mais algumas questões para o formulário  não ficar
%		tão vazio
%	colocar o pontilhado no lugar certo (para isso tem que ser
%		formulada a parte final destacável)
%	parte destacável, que contém:
%		agradecimento
%		conatos
%		logo do sancaLUG
%%%

\end{document}
